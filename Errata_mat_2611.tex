\documentclass[11pt]{article}
\usepackage{tikz}
\usepackage{array}
\usepackage{amsfonts}
\usepackage{amsmath}
\usepackage{amssymb}
\usepackage{amsthm}
\usepackage{physics}
\usepackage{algorithm}
\usepackage{algorithmic}
\newcommand{\egaltext}[1]{\ensuremath{\stackrel{\text{#1}}{=}}}
\newcommand{\Mod}[1]{\ (\mathrm{mod}\ #1)}
\newtheorem*{remark}{Remarque}
\begin{document}

\title{Errata mat 2611}
\author{Xavier Généreux}
\maketitle


\subsection*{1.0.1}
la théorie \textbf{des} anneaux

\subsection*{1.0.2}
il existe \textbf{des} entiers positifs

\subsection*{1.0.3}

La suite ne semble pas \textbf{avoir} de régularité.

\subsection*{1.6.3}
Du \textbf{Lemme} 1.1 \dots????


\subsection*{2.1.5}
 ...et ses \textbf{opérations d'}addition et \textbf{de} multiplication...Les anneaux possèdent souvent d'autres propriétés utiles ...

\subsection*{Exemple 2.4.4}
...Une notion importante dans les anneaux \textbf{polynomiaux}...

\subsection*{2.4.1}
..., alors \textbf{la} pair (a, b) \textbf{vit} dans...

\subsection*{Exemple 2.19}
$$\mathbb{Q}(x) = \left\lbrace \frac{f(x)}{g(x)}:f(x),g(x)\in\pmb{\mathbb{Z}[x]},g(x) \neq 0\right\rbrace$$

\dots son corps des fractions en \textbf{termes} des opérations\dots

\subsection*{4.7.-1}
Vu que $\pmb{a_js_jb_j = (a_jb_j)s_j}$ par commutativité, 
\\
 Finalement, vu que $\pmb{a_js_j = a_j \cdot s_j} \cdot 1$

\subsection*{4.8.4}

 les idéaux \textbf{propres} sont de type fini sont appelés \textbf{noethériens}.


 	
\end{document} 
